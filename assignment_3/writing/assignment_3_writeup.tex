\PassOptionsToPackage{unicode=true}{hyperref} % options for packages loaded elsewhere
\PassOptionsToPackage{hyphens}{url}
%
\documentclass[]{article}
\usepackage{lmodern}
\usepackage{amssymb,amsmath}
\usepackage{ifxetex,ifluatex}
\usepackage{fixltx2e} % provides \textsubscript
\ifnum 0\ifxetex 1\fi\ifluatex 1\fi=0 % if pdftex
  \usepackage[T1]{fontenc}
  \usepackage[utf8]{inputenc}
  \usepackage{textcomp} % provides euro and other symbols
\else % if luatex or xelatex
  \usepackage{unicode-math}
  \defaultfontfeatures{Ligatures=TeX,Scale=MatchLowercase}
\fi
% use upquote if available, for straight quotes in verbatim environments
\IfFileExists{upquote.sty}{\usepackage{upquote}}{}
% use microtype if available
\IfFileExists{microtype.sty}{%
\usepackage[]{microtype}
\UseMicrotypeSet[protrusion]{basicmath} % disable protrusion for tt fonts
}{}
\IfFileExists{parskip.sty}{%
\usepackage{parskip}
}{% else
\setlength{\parindent}{0pt}
\setlength{\parskip}{6pt plus 2pt minus 1pt}
}
\usepackage{hyperref}
\hypersetup{
            pdftitle={Assignment \# 3},
            pdfauthor={Lily Shapiro, Mahalia Clark, \& Thomas O'Leary},
            pdfborder={0 0 0},
            breaklinks=true}
\urlstyle{same}  % don't use monospace font for urls
\usepackage[left=2cm,right=2cm,top=2cm,bottom=2cm]{geometry}
\usepackage{graphicx,grffile}
\makeatletter
\def\maxwidth{\ifdim\Gin@nat@width>\linewidth\linewidth\else\Gin@nat@width\fi}
\def\maxheight{\ifdim\Gin@nat@height>\textheight\textheight\else\Gin@nat@height\fi}
\makeatother
% Scale images if necessary, so that they will not overflow the page
% margins by default, and it is still possible to overwrite the defaults
% using explicit options in \includegraphics[width, height, ...]{}
\setkeys{Gin}{width=\maxwidth,height=\maxheight,keepaspectratio}
\setlength{\emergencystretch}{3em}  % prevent overfull lines
\providecommand{\tightlist}{%
  \setlength{\itemsep}{0pt}\setlength{\parskip}{0pt}}
\setcounter{secnumdepth}{0}
% Redefines (sub)paragraphs to behave more like sections
\ifx\paragraph\undefined\else
\let\oldparagraph\paragraph
\renewcommand{\paragraph}[1]{\oldparagraph{#1}\mbox{}}
\fi
\ifx\subparagraph\undefined\else
\let\oldsubparagraph\subparagraph
\renewcommand{\subparagraph}[1]{\oldsubparagraph{#1}\mbox{}}
\fi

% set default figure placement to htbp
\makeatletter
\def\fps@figure{htbp}
\makeatother

\usepackage{etoolbox}
\makeatletter
\providecommand{\subtitle}[1]{% add subtitle to \maketitle
  \apptocmd{\@title}{\par {\large #1 \par}}{}{}
}
\makeatother

\title{Assignment \# 3}
\providecommand{\subtitle}[1]{}
\subtitle{Modeling Complex Systems (CS/CSYS 302)}
\author{Lily Shapiro, Mahalia Clark, \& Thomas O'Leary}
\date{}

\begin{document}
\maketitle

\hypertarget{question-1-the-cascade-model}{%
\section{Question 1: The cascade
model}\label{question-1-the-cascade-model}}

\hypertarget{part-a}{%
\subsection{Part A}\label{part-a}}

This network will always be acyclic because if we pick a random node,
\(i\), and step along one of its outward edges chosen at random to
another node \(i'\), and keep stepping along random outward edges from
node to node, we can only step to nodes whose value of i is smaller,
never larger. So wherever we start, we will always end up at a smaller
value of \(i\). We can never step from a node with a small value of
\(i\) back to a node with a larger value of \(i\), so we can never
return to our starting point to complete a loop.

\hypertarget{part-b}{%
\subsection{Part B}\label{part-b}}

To calculate the average in-degree of vertex \(i\) we can think about
all the nodes that have an index \(> i\), of which there are \(N - i\).
Each of these could be connected to node \(i\) with a probability \(p\),
so the average in-degree of vertex \(i\) is \((N-i)*p\). Conversely, to
get the average out-degree of vertex \(i\) we can think about the number
of nodes with index \(< i\), of which there are \((i-1)\), so the
average out-degree of vertex \(i\) is \((i-1)*p\).

\hypertarget{part-c}{%
\subsection{Part C}\label{part-c}}

To calculate the expected number of edges that run from nodes \(i' > i\)
(`big nodes') to nodes \(i' \leq i\) (`small nodes') we can begin by
counting the number of big nodes and the number of small nodes. Then we
can think about the expected number of connections between them if nodes
in one group are connected to nodes in the other group with probability
p.~There are \(N–i\) nodes with \(i' > i\) (big nodes) and i nodes with
\(i' \leq i\) (small nodes). Any individual node among the \(N–i\) big
nodes could have up to i edges running to small nodes, one to each of
them, and if each of these edges exists with probability p, we would
expect any single big node to have \(i*p\) edges to small nodes. If we
sum these up over all the big nodes, we get a total of \((N–i)*i*p\)
edges from big nodes to small nodes, or \((N*i–i^2)*p\) edges.

\hypertarget{part-d}{%
\subsection{Part D}\label{part-d}}

Assuming N is even, what are the largest and smallest values of the
quantity calculated in c) and where do they occur (in terms of \(i\))?
If we plotted \((N*i – i2)*p\) over \(i\), we would get a
downward-facing parabola with roots at \(i = 0\) and \(i = N\). Since
\(i\) can only have values from \(1\) to \(N\), it cannot equal zero, so
the smallest value of the function is \(0\) at \(i = N\). The largest
value of the function occurs at the top of the parabola, which is found
halfway between the roots, at \(i = frac{N}{2}\), where the function has
a value of \(frac{N^2}/{4}\).

\hypertarget{question-2}{%
\section{Question 2}\label{question-2}}

\hypertarget{question-3}{%
\section{Question 3}\label{question-3}}

Our network represents the 48 contiguous US states in addition to the
District of Columbia (49 nodes total). Knowing that question 4 revolved
around analyzing the voter model and the capacity of a network to reach
consensus (i.e.~the same state of all nodes), we thought it pertinent to
have our chosen network directly correspond to this, thereby choosing a
basic U.S. map. The appearance of the network itself loosely matches
with the geographic orientation of states and subregions within the U.S
and a view may decipher patterns corresponding to known locations of
states in such a network structure, although there are obvious
departures as the network needs not necessarily exactly correspond.
Clearly the nodes in this network are representative of US states, as
labeled, with the edges representing the geographic borders shared
between states. Maine is the only U.S. state connected by a single edge,
sharing a land border only with New Hampshire (the beautiful seacoast of
NH prevents it from coming into contact with Massachusetts). Other
states are far more connected, particularly Missouri and Tennessee which
have 8 edges apiece. The average degree (or average number of edges
shared between states) is 4.28, a logical output as most states share
about 4 to 5 borders with other states.

In Fig() we have displayed the nodes according to
betweenness-centrality, where it is obvious that Missouri is the most
central state within the network, that is, it most often acts as a
connection between shortest path lengths amongst other nodes. New York
also serves as an apparently important central state, acting as an
apparent bottleneck, cutting off New England from the rest of the
country. States that have high betweenness centrality likely have larger
influence on network dynamics, possibly including how and if states will
reach a consensus within the voter model.

The graph is moderately clustered, with a clustering coefficient of
\(0.507\), which is in accordance with the relatively regular geographic
structure of states on the map. Assessing the modularity of the network
reveals 6 distinct communities (Fig:) with an associated value of
\(0.579\). These more densely connected communities may influence each
other more than those outside of these communities.

\hypertarget{question-4}{%
\section{Question 4}\label{question-4}}

\hypertarget{part-a-1}{%
\subsection{Part A}\label{part-a-1}}

Using the network you chose in the previous problem, implement the voter
model and run it on your network using some initial conditions
(i.e.~50\% red and 50\% blue randomly distributed). Averaged over a few
runs, does the network tend reach consensus? If so, how fast? Try a few
different initial conditions.

\hypertarget{intial-conditions}{%
\subsubsection{Intial Conditions}\label{intial-conditions}}

\includegraphics{assignment_3_writeup_files/figure-latex/unnamed-chunk-1-1.pdf}

\textbf{Fig. XXX:} Graphical representation of the intiail conditions
for the networks for each of the initial conditions that we tested, 2016
Presidential Map (top), Degree-based map (middle), and a Random Map
(bottom).

\hypertarget{voter-model-runs}{%
\section{Voter Model Runs}\label{voter-model-runs}}

\includegraphics{assignment_3_writeup_files/figure-latex/unnamed-chunk-2-1.pdf}

\hypertarget{part-b-1}{%
\subsection{Part B}\label{part-b-1}}

Now, find the degree distribution of your network and run (or imagine)
the voter model using a resulting configuration model network. How do
you expect the dynamics to compare to the real network? Explain with
simulation results or in words.

\includegraphics{assignment_3_writeup_files/figure-latex/unnamed-chunk-3-1.pdf}

\textbf{Fig. XXX:} Degree distribution of the United States Map Network

\end{document}
